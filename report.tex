\documentclass[%
  a4paper,fontsize=11pt,abstract=on,%
  twoside,BCOR=19mm,% for print version adapt...
  %final % activate for final submission, removes draft status and date
]{scrreprt}
\defaulthyphenchar=127  % make all hyphens additional

% encoding (keep this and use UTF-8)
\usepackage[utf8]{inputenc}
\usepackage[T1]{fontenc}

% select the thesis language
% \usepackage[ngerman]{babel}  % select this for a German thesis
\usepackage[english]{babel}  % select this for an English thesis


% file with common and simple user defs

% set these accordingly
\newcommand{\myThesisType}{Master Project}  % Master Thesis / Seminararbeit
\newcommand{\myName}{Mohammad Sakinini}
\newcommand{\myLocation}{Bonn}
\newcommand{\myTitle}{LitQEval: Measuring the Effectiveness of Literature Search Queries}
\newcommand{\mySubtitle}{}
\newcommand{\myKeywords}{Keywords describing this work}
\newcommand{\myThesisStudyCourse}{Computer Science}
\newcommand{\myThesisStudyCourseGerman}{Informatik}
\newcommand{\myThesisSupervisorExternal}{Philipp Baaden, Fraunhofer INT}
\newcommand{\myThesisExternalCompany}{}
\newcommand{\myProf}{Prof.\ Dr.\ Jörn Hees, H-BRS}
\newcommand{\myOtherProf}{Dr.\ Milos Jovanovic, Fraunhofer INT}
\newcommand{\myThesisSubDate}{\today}  % Put your submission date here, the Draft date will disappear when you set the "final" documentclass option in the thesis.tex


% you can define simple other commands here...
% if you want to do something more fancy or import other packages
% you probably want to just modify FrontBackmatter/preabmle.tex


% file taking care of imports, setup etc.
% some basic setting and imports
\usepackage{etoolbox} % extended toolbox for macros
\usepackage{xspace} % to get the spacing after macros right

\usepackage{csquotes} % dealing with "quotes" easily/properly
\MakeOuterQuote{"}

\usepackage[iso,\languagename]{isodate} % e.g., 2023-02-28
\usepackage{datetime2}
\usepackage{iflang} % \IfLanguageName{ngerman}{Deutsche Version}{English Version}

% get us an \IfFinal{in case final}{in case not final} command
\makeatletter\@ifclasswith{scrreprt}{final}{
    \newcommand{\IfFinal}[2]{#1}
}{
    \newcommand{\IfFinal}[2]{#2}
}\makeatother

% in case something uses these
\title{\myTitle}
\author{\myName}
\date{\IfFinal{\myThesisSubDate}{\today}}



% basic packages
%\usepackage{geometry} % clashes with BCOR
\usepackage[dvipsnames]{xcolor}
\usepackage{tcolorbox}
\usepackage{graphicx}
\usepackage{svg}
\usepackage{tikz}
\usetikzlibrary{calc}

% floats: tables, (sub)figures, and captions
\usepackage{tabularx} % better tables
  \setlength{\extrarowheight}{2pt} % increase table row height
\newcommand{\tableheadline}[1]{\multicolumn{1}{c}{\spacedlowsmallcaps{#1}}}
\newcommand{\myfloatalign}{\centering} % to be used with each float for alignment
\usepackage{caption}
\usepackage{makecell}
\usepackage{multirow} % Required to enable merging of columns/rows in a table
\usepackage{longtable}
\captionsetup{font=small} % format=hang,
\usepackage{subfig}
%\usepackage{varwidth}  % like minipage but "up to width"
%\usepackage{afterpage}
%\usepackage[below]{placeins}
\usepackage{rotating} % Required to display sideways tables/figures

% math packages
% \PassOptionsToPackage{fleqn}{amsmath}  % math environments and more by the AMS
\usepackage{amsmath}
\usepackage{amsfonts}
\usepackage{amssymb}
%\usepackage{amsthm}
%\usepackage{marginnote}
%\usepackage{mathtools}
%\usepackage{complexity}
%\usepackage{siunitx}
%\usepackage{bm} % Required for showing math symbols in boldface


% math formulae:
\renewcommand{\vec}[1]{\mathbf{#1}}
\newcommand{\lO}{\mathcal{O}}
\DeclareMathOperator*{\avg}{avg}



\usepackage{hyperref}
\definecolor{webgreen}{rgb}{0,.5,0}  % as in classicthesis
\definecolor{webbrown}{rgb}{.6,0,0}
\hypersetup{
  % draft, % hyperref's draft mode, for printing see below
  % uncomment the following line if you want to have black links (e.g., for printing)
  %colorlinks=false, linktocpage=false, pdfstartpage=1, pdfstartview=FitV, pdfborder={0 0 0},%
  %urlcolor=Black, linkcolor=Black, citecolor=Black, %pagecolor=Black,%
  colorlinks=true, linktocpage=true, pdfstartpage=1, pdfstartview=FitV,%
  urlcolor=webbrown, linkcolor=RoyalBlue, citecolor=webgreen, %pagecolor=RoyalBlue,%
  breaklinks=true, pdfpagemode=UseNone, pageanchor=true, pdfpagemode=UseOutlines,%
  plainpages=false, bookmarksnumbered, bookmarksopen=true, bookmarksopenlevel=1,%
  hypertexnames=true, pdfhighlight=/O,%nesting=true,%frenchlinks,%
  pdftitle={\myTitle},
  pdfsubject={\mySubtitle},
  pdfauthor={\myName},
  pdfcreator={pdfLaTeX},
  pdfkeywords={\myKeywords}
}


% good for \autoref{}
\addto\extrasenglish{% only works with \usepackage[english]{babel}
  \renewcommand*{\partautorefname}{Part}%
  \renewcommand*{\chapterautorefname}{Chapter}%
  \renewcommand*{\sectionautorefname}{Section}%
  \renewcommand*{\subsectionautorefname}{Section}%
  \renewcommand*{\subsubsectionautorefname}{Section}%
  \renewcommand*{\paragraphautorefname}{Paragraph}%
  \renewcommand*{\subparagraphautorefname}{Paragraph}%
  \renewcommand*{\appendixautorefname}{Appendix}%
  \renewcommand*{\pageautorefname}{Page}%
  \renewcommand*{\figureautorefname}{Figure}%
  \renewcommand*{\tableautorefname}{Table}%
  \renewcommand*{\footnoteautorefname}{Footnote}%
  \renewcommand*{\FancyVerbLineautorefname}{Line}%
  \renewcommand*{\theoremautorefname}{Theorem}%
  \renewcommand*{\equationautorefname}{Equation}%
  \renewcommand*{\itemautorefname}{Item}%
}%
\addto\extrasngerman{%
  \renewcommand*{\partautorefname}{Teil}%
  \renewcommand*{\chapterautorefname}{Kapitel}%
  \renewcommand*{\sectionautorefname}{Abschnitt}%
  \renewcommand*{\subsectionautorefname}{Abschnitt}%
  \renewcommand*{\subsubsectionautorefname}{Abschnitt}%
  \renewcommand*{\paragraphautorefname}{Absatz}%
  \renewcommand*{\subparagraphautorefname}{Absatz}%
  \renewcommand*{\appendixautorefname}{Anhang}%
  \renewcommand*{\pageautorefname}{Seite}%
  \renewcommand*{\figureautorefname}{Abbildung}%
  \renewcommand*{\tableautorefname}{Tabelle}%
  \renewcommand*{\footnoteautorefname}{Fußnote}%
  \renewcommand*{\FancyVerbLineautorefname}{Zeile}%
  \renewcommand*{\theoremautorefname}{Theorem}%
  \renewcommand*{\equationautorefname}{Gleichung}%
  \renewcommand*{\itemautorefname}{Punkt}%
}%
\providecommand{\subfigureautorefname}{\figureautorefname}





% ***************
% biblatex setup
% ***************
\PassOptionsToPackage{%
  backend=biber,%bibencoding=utf8, %instead of bibtex
  safeinputenc=true,
  %backend=bibtex8,bibencoding=ascii,%
  language=auto,%
  style=numeric-comp,%
  %style=authoryear-comp, % Author 1999, 2010
  %style=iso-authoryear,%
  %bibstyle=authoryear,dashed=false, % dashed: substitute rep. author with ---
  giveninits=true, % abbreviate firstnames
  sorting=nyt, % name, year, title
  maxbibnames=10, % default: 3, et al.
  backref=true,%
  %isbn=false,
  natbib=true % natbib compatibility mode (\citep and \citet still work)
}{biblatex}
    \usepackage{biblatex}

% enable linebreaks in bib urls
\setcounter{biburlnumpenalty}{7000}
\setcounter{biburlucpenalty}{7000}
\setcounter{biburllcpenalty}{8000}

% add actual bibtex file
\addbibresource{literature.bib}

% KOMA settings for bib
\KOMAoptions{bibliography=totoc} % totocnumbered}
\KOMAoptions{listof=totoc}



% Setup for code listings
% \usepackage{listings}
% \usepackage{scrhack}  % useful workarounds for some package (e.g., listings) incompat with newer koma
% %\lstset{emph={trueIndex,root},emphstyle=\color{BlueViolet}}%\underbar} % for special keywords
% \lstset{%
%     language=Python,%[LaTeX]Tex,%C++,
%     morekeywords={PassOptionsToPackage,selectlanguage},
%     keywordstyle=\color{RoyalBlue},%\bfseries,
%     basicstyle=\small\ttfamily,
%     %identifierstyle=\color{NavyBlue},
%     commentstyle=\color{Green}\ttfamily,
%     stringstyle=\rmfamily,
%     numbers=none,%left,%
%     numberstyle=\scriptsize,%\tiny
%     stepnumber=5,
%     numbersep=8pt,
%     showstringspaces=false,
%     breaklines=true,
%     % frameround=ftff,
%     frame=single,
%     belowcaptionskip=.75\baselineskip
%     %frame=L
% }
%\usepackage{minted}  % listings alternative
%\usepackage{alltt}  % maybe also useful




% Layout stuff
% (recommendation: leave as is, unless forced by your supervisors)

% line spacing
% \usepackage[singlespacing]{setspace}

%% Fontsizes
% \RedeclareSectionCommand[
%     afterskip=0pt,
%     beforeskip=0pt
% ]{chapter}
% \setkomafont{chapter}{\large}
% \RedeclareSectionCommand[
%     afterskip=0pt,
%     beforeskip=4pt,
%     runin=false,
% ]{section}
% \setkomafont{section}{\normalsize}
% \RedeclareSectionCommand[
%     afterskip=0pt,
%     beforeskip=3pt,
%     runin=false
% ]{subsection}
% \setkomafont{subsection}{\normalsize}
% \setkomafont{subsubsection}{\normalsize}
% \setkomafont{caption}{\small}

%% Paragraph
% \KOMAoptions{parskip=half}


% Header & Footer
\usepackage[headsepline, automark]{scrlayer-scrpage} % plainheadsepline
\pagestyle{scrheadings}
% \clearpairofpagestyles
\addtokomafont{pagehead}{\small\normalfont}
% \ihead*{\myTitle}
% \ofoot*{\pagemark}



% other useful packages:
% \usepackage{pgfgantt}

% To-dos
% \usepackage{snaptodo}
% \usepackage[disable]{todonotes}  % disable % to just get rid of todonotes (for review)
\usepackage[obeyFinal]{todonotes}  % doc final will also disable remaining todonotes
%\usepackage{changebar}
\usepackage{verbatim}
\usepackage{lipsum}

\pdfminorversion=6



\begin{document}
\pagenumbering{roman}
\begin{titlepage}
    \thispagestyle{empty}

    % BG Logo (according to H-BRS corporate design guide)
    \begin{tikzpicture}[remember picture, overlay]
        \filldraw[color=black!5]($(current page.north west) + (0,-12)$) circle (\paperwidth * 5 / 14);
        \filldraw[color=white]($(current page.north west) + (0,-12)$) circle (\paperwidth * 3 / 14);
        \filldraw[color=black!5]($(current page.north east) + (0,-12)$) circle (\paperwidth * 5 / 14);
    \end{tikzpicture}

    \vspace{-2cm}
    {
    \includegraphics[width=9.2cm]{FrontBackmatter/H-BRS-Logo mit Fachbereich Informatik_DE_EN.pdf}
    % \includegraphics{FrontBackmatter/H-BRS_Logo_A4.pdf}

    % \vspace{1em}
    % \noindent
    % \hspace{3.614cm}
    % \parbox{10cm}{
    % \sffamily
    % \large
    % \textbf{Fachbereich Informatik}\\
    % Department of Computer Science
    % }
    }

    \vfill
    \begin{center}

        \Large\normalfont\myThesisType\\[1ex]
        \large\normalfont\IfLanguageName{ngerman}{\myThesisStudyCourseGerman}{\myThesisStudyCourse}
        
        \vfill
        
        \Huge\normalfont\myTitle\\[1ex]
        \ifdefempty{\mySubtitle}{}{\huge\normalfont\mySubtitle}

        \vfill

        \Large\myName
        

        \vfill
        \vfill

        \large
        \begin{tabular}{ll}
            \ifdefempty{\myThesisSupervisorExternal}{}{%
                \IfLanguageName{ngerman}{Betreuer}{Supervisor}
                    & \myThesisSupervisorExternal \\}
            \ifdefempty{\myThesisExternalCompany}{}{%
                    & \myThesisExternalCompany    \\[1ex]}
            \IfLanguageName{ngerman}{Erstprüfer}{Examiner 1}
                & \myProf    \\
            \IfLanguageName{ngerman}{Zweitprüfer}{Examiner 2}
                & \myOtherProf   \\[2em]
            \IfFinal{}{\IfLanguageName{ngerman}{Vorabversion vom}{Draft as of} & \DTMnow\\
            & \small \IfLanguageName{ngerman}{(Zur Abgabe: \texttt{final} Option in thesis.tex setzen!)}{(For submission: set \texttt{final} option in thesis.tex!)}\\}
            \IfFinal{\IfLanguageName{ngerman}{Eingereicht am}{Submitted on} }{\IfLanguageName{ngerman}{Einzureichen am}{To be submitted on}}
                & \myThesisSubDate
        \end{tabular}
    \end{center}
\end{titlepage}


% feel free to out-source longer chapter / sections into own files
% e.g., with as above with \include{Content/...}

\begin{abstract}
This work is based on a larger initiative known as the Search Query Writer (SQW), an internal tool developed at Fraunhofer INT to aid scientific researchers in creating comprehensive literature search queries. These queries are intended to provide researchers with a strong starting point in a topic area they may have limited knowledge about. 

The current state of the SQW tool presents a key challenge: the absence of a mechanism to evaluate the quality of the generated queries. As a result, the evaluation has so far been conducted subjectively. This project aims to address this issue by introducing a dataset that contains publications deemed relevant to specific topics. Additionally, it introduces several metrics to account for different aspects of query evaluation, given the complexity of the task.

\textbf{(Explain performed experiments after completing them)}



\end{abstract}

\cleardoublepage
\tableofcontents
% if your professor wants them...
% \listoffigures
% \listoftables

\cleardoublepage
\setcounter{page}{1}
\pagenumbering{arabic}


\chapter{Introduction}\label{ch:intro}

The Fraunhofer Institute for Technological Trend Analysis (INT)\footnote{\url{https://www.int.fraunhofer.de/}} specializes in conducting technology foresight, tackling tasks and research questions across a diverse array of fields. These challenges often necessitate systematic and scientifically sound approaches, even when prior knowledge in the domain is sparse. To address this recurring need, a tool that assists researchers by generating effective search queries as entry points into unfamiliar subject areas becomes essential. For instance, when faced with a specific technological research question, the process typically begins with a thorough literature search using databases like Dimensions \autocite{Hook2018}, Web of Science\footnote{\url{https://clarivate.com/}}, and Scopus\footnote{\url{https://www.elsevier.com/}}. This step involves crafting a precise search query to locate relevant studies, enabling researchers to deliver foresight grounded in scientific evidence.

To address this, several internal tools such as Topic Modeling and Grants Analytics have been developed to analyze large volumes of scientific data from sources like Dimensions.ai and Web of Science. The rise of Large Language Models (LLMs) has further enhanced the appeal and accessibility of automation across numerous domains, including scientific research, spanning from idea generation and experimental iteration to paper composition \autocite{lu2024aiscientistfullyautomated}.

In the realm of search queries, the main focus has been on text-to-SQL \autocite{dong2023c3}, where an LLM is prompted via natural language to generate a precise and valid SQL query. However, to our knowledge, there has been limited effort dedicated to the development of text-to-literature search queries. Thus this work introduces an evaluation pipeline and curates a dataset designed to help address this gap, with a particular focus on enhancing the evaluation the quality of literature search queries using a novel approach called \textit{Semantic Precision}.

The evaluation of literature search queries is inherently complex due to several factors. One major challenge is the tendency to retrieve an overwhelming number of publications. In the end, only a small subset is considered relevant. Another challenge stems from the different objectives of the queries constructed. For example, Systematic Literature Reviews (SLRs) aim to identify every potentially relevant publication through exhaustive search strategies. In contrast, Bibliometric Analyses (BAs) focus on defining a large, relevant set of publications to be quantitatively evaluated. A common problem in both approaches is the initial identification of relevant publications within a large retrieved dataset.

To address this issue, we introduce Semantic Precision: a method for assessing the relevance of publications based on their semantic similarity of the title and abstract to a defined set of core publications. This approach is the basis for the construction of an adjusted $F_\beta$ metric, which includes the recall, the semantic precision, and an additional decay factor. The decay factor allows researchers to tailor the evaluation according to the specific intent of the literature review, whether it aligns with the comprehensive goals of SLRs or the quantitative focus of BAs. By accounting for these elements, our method provides a highly refined and focused framework for evaluating the effectiveness of search queries.

\section{Motivation}
The SQW tool is currently under development by Fraunhofer INT and has generated interest among researchers internally. However, a primary challenge researchers face after testing earlier versions is evaluating the quality of the generated queries. Initially, we considered gathering human feedback from users by requesting them to rate the generated query on a scale of 0 to 5. While this approach could be useful for fine-tuning the underlying model, the quantity of feedback has so far been limited and remains subjective. This is especially problematic because the tool’s purpose is to generate queries for researchers who are new to a given topic. Consequently, if the query quality is poor, the researcher may not immediately recognize this.

Identifying suitable evaluation metrics and datasets to assess the quality of the generated queries is a complex task, which forms the basis of this master's project. The project’s objective is to find a robust solution for assessing the quality of literature search queries, enabling the further development of the SQW tool to provide more accurate results and improve productivity through the integration of LLMs.


\section{Research Question}\label{sec:researchQuestions}

Our work is driven by a central research question that guides both the curation of the dataset and the formulation of metrics for evaluating the effectiveness of the generated queries. The root of this question is the following hypothesis: Given that we know important publications in a given field, referred to as Core Publications (CPs), we can design metrics to evaluate the performance of search queries based on their ability to balance relevance and specificity. This leads to the following research questions: \textbf{Which metric can effectively penalize the generation of excessively large queries that achieve high recall at the cost of precision?} By addressing this question, we aim to develop an evaluation framework that discourages the trivial exploitation of large query sizes and instead rewards meaningful query design that aligns with the intent and context of the literature search.


\section{Structure of this Work}\label{sec:structure}
The remainder of this work is structured as follows:

After this introduction, we will first focus on the foundations in \autoref{ch:foundations}, where the SQW tool will be briefly explained, primarily focusing on the format of the input and the stages that the SQW consists of. Subsequently, we will explore related works in \autoref{sec:relwork} and review the currently available datasets, explaining why they are not suitable for our specific use case.

Next, we introduce our framework, which consists of two main components: the curated dataset in \autoref{sec:dataset} and the evaluation metrics in \autoref{sec:eval-metrics}. In the dataset section, we explain how the data was collected and perform a dataset analysis to gain deeper insights into its characteristics. In the evaluation metrics section, we present the metrics developed to assess the performance of literature search queries.

Following this, we present an evaluation of the framework and showcase the results in \autoref{ch:eval}. This chapter begins with a description of the conducted experiments, where two types of queries are used: those written for systematic literature reviews (SLR) and those generated by the SQW, which are then used for evaluation.

Finally, we conclude this work with a summary of the main contributions and provide an outlook on future directions in \autoref{ch:conclusion}.







\chapter{Foundations}\label{ch:foundations}
\lipsum[1]

\section{Basics / State of the Art}\label{sec:basics}
Pay attention to citing the right sources, e.g. Knuth's "TAoCP" \cite{DBLP:books/lib/Knuth97}.

\lipsum[1-3]


\section{Common Metrics}\label{sec:metrics}
\lipsum[1-3]
\section{Related Work}\label{sec:relwork}
\lipsum[1-3]



\chapter{Own Approach}\label{ch:ownApproach}
\lipsum[1]
\section{Idea / Overview}
\lipsum[1-3]
\section{Component 1}
\lipsum[1-3]
\section{Component 2}
\lipsum[1-3]
\section{Optionals}
\lipsum[1-3]



\chapter{Evaluation}\label{ch:eval}
\lipsum[1]
\section{Experimental Setup}
\lipsum[1-3]
\section{Results}
\lipsum[1-3]
\section{Discussion}
\lipsum[1-3]



\chapter{Conclusion}\label{ch:conclusion}
\lipsum[1]
\section{Summary and Contributions}
\lipsum[1-3]
\section{Outlook}
\lipsum[1-3]



% if appendix is more than just a few pages long, maybe put this
% after bib and declaration... ask your supervisor
% if there's none, comment out / remove
\appendix
\chapter{Appendix}
\section{Further Details on Something}
\lipsum



\printbibliography

\begingroup
\addcontentsline{toc}{chapter}{\IfLanguageName{ngerman}{Erklärung}{Declaration}}
\chapter*{\IfLanguageName{ngerman}{Erklärung}{Declaration}}
% \thispagestyle{empty}
\IfLanguageName{ngerman}{
Ich versichere, die von mir vorgelegte Arbeit selbstständig verfasst zu haben.
Alle Stellen, die wörtlich oder sinngemäß aus veröffentlichten oder nicht veröffentlichten Arbeiten Dritter entnommen sind, habe ich als solche kenntlich gemacht.
Sämtliche Quellen und Hilfsmittel, die ich für die Arbeit benutzt habe, sind angegeben.

% adapt if tools were used
% remove if no tools have been used
(Es folgen Formulierungsbeispiele, die Sie im Sinne der Transparenz an Ihre Arbeit anpassen müssen. Über die Zulässigkeit von Hilfsmitteln sollten Sie natürlich im Vorfeld mit Ihrem Betreuer diskutiert haben.)
Insb. wurden zur Erstellung dieser Arbeit auch folgende KI Systeme eingesetzt:
\begin{itemize}
    \item ChatGPT in Version ... wurde zur initialen Ausformulierung basierend auf von mir vorgegebenen Stichpunkten in den Kapiteln ... / der gesamten Arbeit eingesetzt.
    \item ChatGPT wurde zu folgenden Themen befragt: ... / zur Generierung von Ideen bzgl. ... / Strukturierung von ... genutzt / zur Konzeption des Systems ... genutzt.
    
    Der Wortlaut der Dialoge, sowie die verwendete Version wurde im Anhang der Arbeit dokumentiert. Genutzte Passagen sind im Text als solche gekennzeichnet.
    \item ChatGPT wurde zur Erstellung von Quellcode für ... genutzt.
    
    Der Wortlaut der Dialoge, sowie die verwendete Version wurde im Anhang der Arbeit dokumentiert. Die Verwendung ist im Kopf der jeweiligen Quelldatei / Klasse / Methode / Teile kenntlich gemacht.
    \item Copilot in Version ... wurde zur Erstellung von Quellcode / Auto-Complete für ... genutzt. Die Verwendung ist im Kopf der jeweiligen Quelldatei / Klasse / Methode / Teile kenntlich gemacht.
\end{itemize}
Mir ist bewusst, dass von KI Systemen generierte Inhalte das sorgfältige wissenschaftliche Arbeiten nicht ersetzen, weshalb sämtliche derartige Inhalte durch mich kritisch überprüft und finalisiert wurden.

Die Arbeit war mit gleichem Inhalt bzw. in wesentlichen Teilen noch nicht Gegenstand einer anderen Prüfung.
}{
I declare that I have written this work by myself.
I have identified as such all passages taken verbatim or in meaning from published or unpublished works by third parties.
All sources and aids that I have used for the work are indicated.

% adapt if tools were used
% remove if no tools have been used
(Example formulations follow, which you must adapt to your work for the sake of transparency. Of course, you should have discussed about the acceptability of such aids with your supervisor in advance.)
In particular, the following AI systems were also used to create this work:
\begin{itemize}
    \item ChatGPT in version ... was used for the initial text drafting based on bullet points given by me in the chapters ... / of the entire work.
    \item ChatGPT was consulted on the following topics: ... / was used to generate ideas regarding ... / for the structuring of ... / for the conception of the system ... .
    
    The wording of the dialogs and the version used were documented in the appendix of this work. Passages used are marked as such in the text.
    \item ChatGPT was used to create source code for ... . The wording of the dialogs and the version used were documented in the appendix of this work. The use is indicated in the header of the respective source file / class / method / parts.
    \item Copilot in version ... was used to create source code / auto-complete for ... . The use is documented in the header of the respective source file / class / method / parts.
\end{itemize}
I am aware that content generated by AI systems is no substitute for careful scientific work, which is why all such generated content has been critically reviewed and finalized by me.

This work has neither been submitted with the same content nor in essential parts to any other examination authority.
}

\bigskip\bigskip
\noindent\textit{\myLocation, \myThesisSubDate}

\smallskip

\begin{flushright}
    \begin{tabular}{m{6cm}}
        \\ \hline
        \centering\myName \\
    \end{tabular}
\end{flushright}

\endgroup






\end{document}
