\documentclass[%
  a4paper,fontsize=11pt,abstract=on,%
  oneside,BCOR=19mm,% for print version adapt...
  %final % activate for final submission, removes draft status and date
]{scrreprt}
\defaulthyphenchar=127  % make all hyphens additional

% encoding (keep this and use UTF-8)
\usepackage[utf8]{inputenc}
\usepackage[T1]{fontenc}

% select the thesis language
% \usepackage[ngerman]{babel}  % select this for a German thesis
\usepackage[english]{babel}  % select this for an English thesis


% file with common and simple user defs

% set these accordingly
\newcommand{\myThesisType}{Master Project}  % Master Thesis / Seminararbeit
\newcommand{\myName}{Mohammad Sakinini}
\newcommand{\myLocation}{Bonn}
\newcommand{\myTitle}{LitQEval: Measuring the Effectiveness of Literature Search Queries}
\newcommand{\mySubtitle}{}
\newcommand{\myKeywords}{Keywords describing this work}
\newcommand{\myThesisStudyCourse}{Computer Science}
\newcommand{\myThesisStudyCourseGerman}{Informatik}
\newcommand{\myThesisSupervisorExternal}{Philipp Baaden, Fraunhofer INT}
\newcommand{\myThesisExternalCompany}{}
\newcommand{\myProf}{Prof.\ Dr.\ Jörn Hees, H-BRS}
\newcommand{\myOtherProf}{Dr.\ Milos Jovanovic, Fraunhofer INT}
\newcommand{\myThesisSubDate}{\today}  % Put your submission date here, the Draft date will disappear when you set the "final" documentclass option in the thesis.tex


% you can define simple other commands here...
% if you want to do something more fancy or import other packages
% you probably want to just modify FrontBackmatter/preabmle.tex


% file taking care of imports, setup etc.
% some basic setting and imports
\usepackage{etoolbox} % extended toolbox for macros
\usepackage{xspace} % to get the spacing after macros right

\usepackage{csquotes} % dealing with "quotes" easily/properly
\MakeOuterQuote{"}

\usepackage[iso,\languagename]{isodate} % e.g., 2023-02-28
\usepackage{datetime2}
\usepackage{iflang} % \IfLanguageName{ngerman}{Deutsche Version}{English Version}

% get us an \IfFinal{in case final}{in case not final} command
\makeatletter\@ifclasswith{scrreprt}{final}{
    \newcommand{\IfFinal}[2]{#1}
}{
    \newcommand{\IfFinal}[2]{#2}
}\makeatother

% in case something uses these
\title{\myTitle}
\author{\myName}
\date{\IfFinal{\myThesisSubDate}{\today}}



% basic packages
%\usepackage{geometry} % clashes with BCOR
\usepackage[dvipsnames]{xcolor}
\usepackage{tcolorbox}
\usepackage{graphicx}
\usepackage{svg}
\usepackage{tikz}
\usetikzlibrary{calc}

% floats: tables, (sub)figures, and captions
\usepackage{tabularx} % better tables
  \setlength{\extrarowheight}{2pt} % increase table row height
\newcommand{\tableheadline}[1]{\multicolumn{1}{c}{\spacedlowsmallcaps{#1}}}
\newcommand{\myfloatalign}{\centering} % to be used with each float for alignment
\usepackage{caption}
\usepackage{makecell}
\usepackage{multirow} % Required to enable merging of columns/rows in a table
\usepackage{longtable}
\captionsetup{font=small} % format=hang,
\usepackage{subfig}
%\usepackage{varwidth}  % like minipage but "up to width"
%\usepackage{afterpage}
%\usepackage[below]{placeins}
\usepackage{rotating} % Required to display sideways tables/figures

% math packages
% \PassOptionsToPackage{fleqn}{amsmath}  % math environments and more by the AMS
\usepackage{amsmath}
\usepackage{amsfonts}
\usepackage{amssymb}
%\usepackage{amsthm}
%\usepackage{marginnote}
%\usepackage{mathtools}
%\usepackage{complexity}
%\usepackage{siunitx}
%\usepackage{bm} % Required for showing math symbols in boldface


% math formulae:
\renewcommand{\vec}[1]{\mathbf{#1}}
\newcommand{\lO}{\mathcal{O}}
\DeclareMathOperator*{\avg}{avg}



\usepackage{hyperref}
\definecolor{webgreen}{rgb}{0,.5,0}  % as in classicthesis
\definecolor{webbrown}{rgb}{.6,0,0}
\hypersetup{
  % draft, % hyperref's draft mode, for printing see below
  % uncomment the following line if you want to have black links (e.g., for printing)
  %colorlinks=false, linktocpage=false, pdfstartpage=1, pdfstartview=FitV, pdfborder={0 0 0},%
  %urlcolor=Black, linkcolor=Black, citecolor=Black, %pagecolor=Black,%
  colorlinks=true, linktocpage=true, pdfstartpage=1, pdfstartview=FitV,%
  urlcolor=webbrown, linkcolor=RoyalBlue, citecolor=webgreen, %pagecolor=RoyalBlue,%
  breaklinks=true, pdfpagemode=UseNone, pageanchor=true, pdfpagemode=UseOutlines,%
  plainpages=false, bookmarksnumbered, bookmarksopen=true, bookmarksopenlevel=1,%
  hypertexnames=true, pdfhighlight=/O,%nesting=true,%frenchlinks,%
  pdftitle={\myTitle},
  pdfsubject={\mySubtitle},
  pdfauthor={\myName},
  pdfcreator={pdfLaTeX},
  pdfkeywords={\myKeywords}
}


% good for \autoref{}
\addto\extrasenglish{% only works with \usepackage[english]{babel}
  \renewcommand*{\partautorefname}{Part}%
  \renewcommand*{\chapterautorefname}{Chapter}%
  \renewcommand*{\sectionautorefname}{Section}%
  \renewcommand*{\subsectionautorefname}{Section}%
  \renewcommand*{\subsubsectionautorefname}{Section}%
  \renewcommand*{\paragraphautorefname}{Paragraph}%
  \renewcommand*{\subparagraphautorefname}{Paragraph}%
  \renewcommand*{\appendixautorefname}{Appendix}%
  \renewcommand*{\pageautorefname}{Page}%
  \renewcommand*{\figureautorefname}{Figure}%
  \renewcommand*{\tableautorefname}{Table}%
  \renewcommand*{\footnoteautorefname}{Footnote}%
  \renewcommand*{\FancyVerbLineautorefname}{Line}%
  \renewcommand*{\theoremautorefname}{Theorem}%
  \renewcommand*{\equationautorefname}{Equation}%
  \renewcommand*{\itemautorefname}{Item}%
}%
\addto\extrasngerman{%
  \renewcommand*{\partautorefname}{Teil}%
  \renewcommand*{\chapterautorefname}{Kapitel}%
  \renewcommand*{\sectionautorefname}{Abschnitt}%
  \renewcommand*{\subsectionautorefname}{Abschnitt}%
  \renewcommand*{\subsubsectionautorefname}{Abschnitt}%
  \renewcommand*{\paragraphautorefname}{Absatz}%
  \renewcommand*{\subparagraphautorefname}{Absatz}%
  \renewcommand*{\appendixautorefname}{Anhang}%
  \renewcommand*{\pageautorefname}{Seite}%
  \renewcommand*{\figureautorefname}{Abbildung}%
  \renewcommand*{\tableautorefname}{Tabelle}%
  \renewcommand*{\footnoteautorefname}{Fußnote}%
  \renewcommand*{\FancyVerbLineautorefname}{Zeile}%
  \renewcommand*{\theoremautorefname}{Theorem}%
  \renewcommand*{\equationautorefname}{Gleichung}%
  \renewcommand*{\itemautorefname}{Punkt}%
}%
\providecommand{\subfigureautorefname}{\figureautorefname}





% ***************
% biblatex setup
% ***************
\PassOptionsToPackage{%
  backend=biber,%bibencoding=utf8, %instead of bibtex
  safeinputenc=true,
  %backend=bibtex8,bibencoding=ascii,%
  language=auto,%
  style=numeric-comp,%
  %style=authoryear-comp, % Author 1999, 2010
  %style=iso-authoryear,%
  %bibstyle=authoryear,dashed=false, % dashed: substitute rep. author with ---
  giveninits=true, % abbreviate firstnames
  sorting=nyt, % name, year, title
  maxbibnames=10, % default: 3, et al.
  backref=true,%
  %isbn=false,
  natbib=true % natbib compatibility mode (\citep and \citet still work)
}{biblatex}
    \usepackage{biblatex}

% enable linebreaks in bib urls
\setcounter{biburlnumpenalty}{7000}
\setcounter{biburlucpenalty}{7000}
\setcounter{biburllcpenalty}{8000}

% add actual bibtex file
\addbibresource{literature.bib}

% KOMA settings for bib
\KOMAoptions{bibliography=totoc} % totocnumbered}
\KOMAoptions{listof=totoc}



% Setup for code listings
% \usepackage{listings}
% \usepackage{scrhack}  % useful workarounds for some package (e.g., listings) incompat with newer koma
% %\lstset{emph={trueIndex,root},emphstyle=\color{BlueViolet}}%\underbar} % for special keywords
% \lstset{%
%     language=Python,%[LaTeX]Tex,%C++,
%     morekeywords={PassOptionsToPackage,selectlanguage},
%     keywordstyle=\color{RoyalBlue},%\bfseries,
%     basicstyle=\small\ttfamily,
%     %identifierstyle=\color{NavyBlue},
%     commentstyle=\color{Green}\ttfamily,
%     stringstyle=\rmfamily,
%     numbers=none,%left,%
%     numberstyle=\scriptsize,%\tiny
%     stepnumber=5,
%     numbersep=8pt,
%     showstringspaces=false,
%     breaklines=true,
%     % frameround=ftff,
%     frame=single,
%     belowcaptionskip=.75\baselineskip
%     %frame=L
% }
%\usepackage{minted}  % listings alternative
%\usepackage{alltt}  % maybe also useful




% Layout stuff
% (recommendation: leave as is, unless forced by your supervisors)

% line spacing
% \usepackage[singlespacing]{setspace}

%% Fontsizes
% \RedeclareSectionCommand[
%     afterskip=0pt,
%     beforeskip=0pt
% ]{chapter}
% \setkomafont{chapter}{\large}
% \RedeclareSectionCommand[
%     afterskip=0pt,
%     beforeskip=4pt,
%     runin=false,
% ]{section}
% \setkomafont{section}{\normalsize}
% \RedeclareSectionCommand[
%     afterskip=0pt,
%     beforeskip=3pt,
%     runin=false
% ]{subsection}
% \setkomafont{subsection}{\normalsize}
% \setkomafont{subsubsection}{\normalsize}
% \setkomafont{caption}{\small}

%% Paragraph
% \KOMAoptions{parskip=half}


% Header & Footer
\usepackage[headsepline, automark]{scrlayer-scrpage} % plainheadsepline
\pagestyle{scrheadings}
% \clearpairofpagestyles
\addtokomafont{pagehead}{\small\normalfont}
% \ihead*{\myTitle}
% \ofoot*{\pagemark}



% other useful packages:
% \usepackage{pgfgantt}

% To-dos
% \usepackage{snaptodo}
% \usepackage[disable]{todonotes}  % disable % to just get rid of todonotes (for review)
\usepackage[obeyFinal]{todonotes}  % doc final will also disable remaining todonotes
%\usepackage{changebar}
\usepackage{verbatim}
\usepackage{lipsum}

\pdfminorversion=6



\begin{document}
\pagenumbering{roman}
\begin{titlepage}
    \thispagestyle{empty}

    % BG Logo (according to H-BRS corporate design guide)
    \begin{tikzpicture}[remember picture, overlay]
        \filldraw[color=black!5]($(current page.north west) + (0,-12)$) circle (\paperwidth * 5 / 14);
        \filldraw[color=white]($(current page.north west) + (0,-12)$) circle (\paperwidth * 3 / 14);
        \filldraw[color=black!5]($(current page.north east) + (0,-12)$) circle (\paperwidth * 5 / 14);
    \end{tikzpicture}

    \vspace{-2cm}
    {
    \includegraphics[width=9.2cm]{FrontBackmatter/H-BRS-Logo mit Fachbereich Informatik_DE_EN.pdf}
    % \includegraphics{FrontBackmatter/H-BRS_Logo_A4.pdf}

    % \vspace{1em}
    % \noindent
    % \hspace{3.614cm}
    % \parbox{10cm}{
    % \sffamily
    % \large
    % \textbf{Fachbereich Informatik}\\
    % Department of Computer Science
    % }
    }

    \vfill
    \begin{center}

        \Large\normalfont\myThesisType\\[1ex]
        \large\normalfont\IfLanguageName{ngerman}{\myThesisStudyCourseGerman}{\myThesisStudyCourse}
        
        \vfill
        
        \Huge\normalfont\myTitle\\[1ex]
        \ifdefempty{\mySubtitle}{}{\huge\normalfont\mySubtitle}

        \vfill

        \Large\myName
        

        \vfill
        \vfill

        \large
        \begin{tabular}{ll}
            \ifdefempty{\myThesisSupervisorExternal}{}{%
                \IfLanguageName{ngerman}{Betreuer}{Supervisor}
                    & \myThesisSupervisorExternal \\}
            \ifdefempty{\myThesisExternalCompany}{}{%
                    & \myThesisExternalCompany    \\[1ex]}
            \IfLanguageName{ngerman}{Erstprüfer}{Examiner 1}
                & \myProf    \\
            \IfLanguageName{ngerman}{Zweitprüfer}{Examiner 2}
                & \myOtherProf   \\[2em]
            \IfFinal{}{\IfLanguageName{ngerman}{Vorabversion vom}{Draft as of} & \DTMnow\\
            & \small \IfLanguageName{ngerman}{(Zur Abgabe: \texttt{final} Option in thesis.tex setzen!)}{(For submission: set \texttt{final} option in thesis.tex!)}\\}
            \IfFinal{\IfLanguageName{ngerman}{Eingereicht am}{Submitted on} }{\IfLanguageName{ngerman}{Einzureichen am}{To be submitted on}}
                & \myThesisSubDate
        \end{tabular}
    \end{center}
\end{titlepage}


\begin{abstract}
This work is based on a larger initiative known as the Search Query Writer (SQW), an internal tool developed at Fraunhofer INT to aid scientific researchers in creating comprehensive literature search queries. These queries are intended to provide researchers with a strong starting point in a topic area they may have limited knowledge about. 

The current state of the SQW tool presents a key challenge: the absence of a mechanism to evaluate the quality of the generated queries. As a result, the evaluation has so far been conducted subjectively. This project aims to address this issue by introducing a dataset that contains publications deemed relevant to specific topics. Additionally, it introduces several metrics to account for different aspects of query evaluation, given the complexity of the task.\textbf{(Explain performed experiments after completing them)}

\end{abstract}

\cleardoublepage
\tableofcontents
% if your professor wants them...
 \listoffigures
% \listoftables

\cleardoublepage
\setcounter{page}{1}
\pagenumbering{arabic}


\chapter{Introduction}\label{ch:intro}

The Fraunhofer Institute for Technological Trend Analysis (INT)\footnote{\url{https://www.int.fraunhofer.de/}} specializes in conducting technology foresight, tackling tasks and research questions across a diverse array of fields. These challenges often necessitate systematic and scientifically sound approaches, even when prior knowledge in the domain is sparse. To address this recurring need, a tool that assists researchers by generating effective search queries as entry points into unfamiliar subject areas becomes essential. For instance, when faced with a specific technological research question, the process typically begins with a thorough literature search using databases like Dimensions \autocite{Hook2018}, Web of Science\footnote{\url{https://clarivate.com/}}, and Scopus\footnote{\url{https://www.elsevier.com/}}. This step involves crafting a precise search query to locate relevant studies, enabling researchers to deliver foresight grounded in scientific evidence.

To address this, several internal tools such as Topic Modeling and Grants Analytics have been developed to analyze large volumes of scientific data from sources like Dimensions.ai and Web of Science. The rise of Large Language Models (LLMs) has further enhanced the appeal and accessibility of automation across numerous domains, including scientific research, spanning from idea generation and experimental iteration to paper composition \autocite{lu2024aiscientistfullyautomated}.

In the realm of search queries, the main focus has been on text-to-SQL \autocite{dong2023c3}, where an LLM is prompted via natural language to generate a precise and valid SQL query. However, to our knowledge, there has been limited effort dedicated to the development of text-to-literature search queries. Thus this work introduces an evaluation pipeline and curates a dataset designed to help address this gap, with a particular focus on enhancing the evaluation the quality of literature search queries using a novel approach called \textit{Semantic Precision}.

The evaluation of literature search queries is inherently complex due to several factors. One major challenge is the tendency to retrieve an overwhelming number of publications. In the end, only a small subset is considered relevant. Another challenge stems from the different objectives of the queries constructed. For example, Systematic Literature Reviews (SLRs) aim to identify every potentially relevant publication through exhaustive search strategies. In contrast, Bibliometric Analyses (BAs) focus on defining a large, relevant set of publications to be quantitatively evaluated. A common problem in both approaches is the initial identification of relevant publications within a large retrieved dataset.

To address this issue, we introduce Semantic Precision: a method for assessing the relevance of publications based on their semantic similarity of the title and abstract to a defined set of core publications. This approach is the basis for the construction of an adjusted $F_\beta$ metric, which includes the recall, the semantic precision, and an additional decay factor. The decay factor allows researchers to tailor the evaluation according to the specific intent of the literature review, whether it aligns with the comprehensive goals of SLRs or the quantitative focus of BAs. By accounting for these elements, our method provides a highly refined and focused framework for evaluating the effectiveness of search queries.

\section{Motivation}
The SQW tool is currently under development by Fraunhofer INT and has generated interest among researchers internally. However, a primary challenge researchers face after testing earlier versions is evaluating the quality of the generated queries. Initially, we considered gathering human feedback from users by requesting them to rate the generated query on a scale of 0 to 5. While this approach could be useful for fine-tuning the underlying model, the quantity of feedback has so far been limited and remains subjective. This is especially problematic because the tool’s purpose is to generate queries for researchers who are new to a given topic. Consequently, if the query quality is poor, the researcher may not immediately recognize this.

Identifying suitable evaluation metrics and datasets to assess the quality of the generated queries is a complex task, which forms the basis of this master's project. The project’s objective is to find a robust solution for assessing the quality of literature search queries, enabling the further development of the SQW tool to provide more accurate results and improve productivity through the integration of LLMs.


\section{Research Question}\label{sec:researchQuestions}

Our work is driven by a central research question that guides both the curation of the dataset and the formulation of metrics for evaluating the effectiveness of the generated queries. The root of this question is the following hypothesis: Given that we know important publications in a given field, referred to as Core Publications (CPs), we can design metrics to evaluate the performance of search queries based on their ability to balance relevance and specificity. This leads to the following research questions: \textbf{Which metric can effectively penalize the generation of excessively large queries that achieve high recall at the cost of precision?} By addressing this question, we aim to develop an evaluation framework that discourages the trivial exploitation of large query sizes and instead rewards meaningful query design that aligns with the intent and context of the literature search.


\section{Structure of this Work}\label{sec:structure}
The remainder of this work is structured as follows:

After this introduction, we will first focus on the foundations in \autoref{ch:foundations}, where the SQW tool will be briefly explained, primarily focusing on the format of the input and the stages that the SQW consists of. Subsequently, we will explore related works in \autoref{sec:relwork} and review the currently available datasets, explaining why they are not suitable for our specific use case.

Next, we introduce our framework, which consists of two main components: the curated dataset in \autoref{sec:dataset} and the evaluation metrics in \autoref{sec:eval-metrics}. In the dataset section, we explain how the data was collected and perform a dataset analysis to gain deeper insights into its characteristics. In the evaluation metrics section, we present the metrics developed to assess the performance of literature search queries.

Following this, we present an evaluation of the framework and showcase the results in \autoref{ch:eval}. This chapter begins with a description of the conducted experiments, where two types of queries are used: those written for systematic literature reviews (SLR) and those generated by the SQW, which are then used for evaluation.

Finally, we conclude this work with a summary of the main contributions and provide an outlook on future directions in \autoref{ch:conclusion}.


{\let\clearpage\relax \chapter{Foundations}\label{ch:foundations}}
In this chapter, we introduce the SQW, a tool designed to generate literature search queries for evaluation in later stages of the project. We begin by outlining the required inputs for the tool, clarifying how users interact with it. Then, we provide an overview of the two main stages of the SQW, explaining the methodology and rationale behind its design.

We also review related work in the field, focusing on approaches that evaluate literature search queries. We assess the strengths and limitations of these methods and discuss how the queries they generate are evaluated using existing datasets. This review provides essential context for understanding the current landscape of literature query evaluation research, which, to our knowledge, has gaps in the area of automatic evaluation. While existing work offers valuable insights into specific aspects of query evaluation, it often overlooks the variety of factors influencing query performance across diverse datasets and domains. This gap highlights the need for further exploration into more robust and automated evaluation frameworks.

\section{Search Query Writer}\label{sec:sqw}
The SQW is a tool based on an LLM, specifically using GPT-4o, to systematically generate literature search queries. The only required input for this tool, which is the main focus of this work, is the \textbf{Topic}. Users are required to provide a topic for generating a search query, irrespective of the scientific field—for example, \textit{Synthetic Biology}. 

Several optional inputs are available to enhance the quality of the generated query, including:
\begin{itemize}
	
\item \textbf{Negative Keywords:} Terms that should be excluded to avoid unwanted results.
\item \textbf{Description:} A description that serves as an alignment mechanism to clarify the task’s intent.
\item \textbf{Modes:} Three selectable modes (Strict, Moderate, Creative) that control the temperature of the LLM to manage the level of randomness in responses.
\item \textbf{Depth:} A parameter that specifies how comprehensively the topic should be analyzed.
\item \textbf{Supporting Documents:} Users can upload a PDF, ideally a survey or overview document on the topic, which helps the tool acquire knowledge about the scientific field and better align with the research intent.

\end{itemize}
These additional inputs are intended to refine and tailor the search query to more closely match the user's research goals, but will not be extensively tested in this work.

To generate a literature search query, we designed the SQW to take a human-like approach, divided into two main steps: \textbf{Knowledge Enrichment} and \textbf{Iterative Scientific Fine-Tuning}.

The objective of the \textbf{Knowledge Enrichment} step is to provide the LLM with contextual information about the research topic. This is achieved by first retrieving information from Wikipedia based on the given topic. Specifically, the first 4,000 characters from the top-\( k \) pages are collected and summarized before being passed into the LLM's memory. ArXiv is queried in a similar manner to gather relevant research content. Additionally, we perform an online search using DuckDuckGo\footnote{\url{https://duckduckgo.com/}}, aggregating results to offer a broader understanding of the topic.

\begin{figure}[!h]
	\centering
	\includegraphics[height=220px, width=400px]{pics/sqw-overview.pdf}
	\caption[Search Query Writer]{A simplified overview of the SQW. The process begins with the Knowledge Enrichment stage, where the model receives input data and sends it to an LLM agent equipped with a suite of tools to gain insights into the topic. Based on this understanding, the model generates a well-structured search query, formatted and executed across multiple dimensions to retrieve a relevant selection of literature. Within this literature set, RAG is applied to identify the most pertinent keywords, which are compiled into an optimized search query. This query can be iteratively refined to enhance overall search quality.}
	\label{fig:sqw-overview}
\end{figure}

To reduce Recency bias \autocite{Deldjoo2024}, which refers to the tendency of the attention mechanism to favor more recent information, each of the steps is conducted in a separate API session, with results stored independently for future use. This approach ensures a that the LLM is biased towards the recently fetched results from prior steps.

The output of this first stage is a list of keywords, typically presented in a transfer-list format, as shown in \autoref{fig:sqw-stage1}. This transfer-list contains two categories: specific and general keywords, along with additional information such as the number of publications found per keyword. The primary goal of this step is to enable the user to assess the relevance of each keyword. If a keyword is deemed too broad, it should be moved to the general list; if it accurately targets the specified topic, it should remain in the specific list. Additionally, the user is required to provide an overarching topic that limits the scope of the general keywords to align more closely with the research intent. The output of this stage will be the queries used for the final evaluation.

\begin{figure}
	\centering
	\includegraphics[height=180px, width=400px]{pics/sqw-stage1.png}
	\caption[SQW Knowledge Enrichment]{A screenshot of the SQW UI after completing the Knowledge Enrichment stage. On the left, a list of keywords is displayed alongside the number of publications associated with each keyword when used as a search term. The keywords on the right-hand side were manually categorized as general and can be roughly assessed by the number of associated publications. To narrow the scope of general keywords, we selected "agriculture" as the overarching topic. The final generated query is displayed and updated interactively as values in the transfer lists are adjusted.}
	\label{fig:sqw-stage1}
\end{figure}


The \textbf{iterative scientific fine-tuning} on the other hand approaches more scientific sources, namely dimensions.ai, which is a literature database that offers quick access to publications across a wide range of journals. The query generated in the earlier stage is then used to prompt dimensions three times, once to retrieve the most cited 1k literature, a second time to retrieve the newest 1k literature, and one last 1k to retrieve the most relevant literature based on their altmetric rating. The results are then combined, and duplicates are removed, yielding a set of publications. For this set, the titles and abstracts are extracted and processed using OpenAI's embedding model. Finally, a simple RAG pipeline is applied to retrieve publications which are then used to generate relevant keywords based on their content.

\section{Related Work}\label{sec:relwork}

SLRs are widely used across various fields, allowing researchers to conduct a comprehensive manual review of scientific topics and identify publications that answer a set of research questions. However, one significant challenge with this approach has been the exponential growth in the number of publications, which makes conducting unbiased reviews increasingly difficult. In the age of technological advancements, we can now use these technologies such as LLMs, Topic Modeling, Semantic Embeddings and much more to assist in investigating topics without the need to manually sift through extensive lists of potentially irrelevant publications. To address this issue, a series of works have been proposed within the Conference and Labs of the Evaluation Forum (CLEF) \autocite{kanoulas2017clef, kanoulas2018clef, kanoulas2019clef}. These works focus on the evaluation of empirical medical research, utilizing a dataset of medical literature. They introduce two primary tasks: Task 1, which involves identifying relevant studies from the PubMed medical database, and Task 2, which assesses the ranking of studies following title and abstract screening. Notably, the evaluation pipeline, along with the dataset and descriptions of these tasks, are publicly accessible on GitHub\footnote{\url{https://github.com/CLEF-TAR/tar/tree/master}}.

LLMs have had significant impact on modern technology, including in scientific research, where they have provided remarkable improvements in efficiency. While the processing efficiency of LLMs is unprecedented, the quality of their output in various domains is still being explored. The work by Wang \autocite{wang2023can} investigated the performance of ChatGPT in generating Boolean search queries for literature reviews. Specifically, it evaluated the effectiveness of ChatGPT in constructing queries for SLRs using different prompting techniques, including zero-shot, few-shot and iterative guided approaches. The evaluation used the CLEF datasets \autocite{kanoulas2017clef, kanoulas2018clef, kanoulas2019clef} and an additional medical dataset containing a collection of seeds \autocite{Wang_2022}. Although the results highlight the limitations of ChatGPT's performance, this work underscores the potential of LLMs to aid literature review, especially when supported by examples or more advanced, structured pipelines.

A broader and more diverse evaluation of the quality of automatically generated literature search queries for SLRs was conducted by Badami \autocite{badami2023adaptive}. In this work, they introduced a pipeline that generates literature search queries based on a given research question and abstracts from previously identified relevant publications. The evaluation was performed against a dataset they constructed, which contains the results of 10 SLRs, including candidate papers, queries used, and relevant papers identified in each review. For example, in the review $SLR_1$, a total of 7,002 candidate papers were retrieved using search query $S$, from which a subset of 59 relevant papers $RP$ was identified. To assess their proposed approach, they compared the generated queries in various settings using recall and precision metrics, benchmarking them against the original search query $S$. The dataset is publicly available on Zenodo\footnote{\url{https://tinyurl.com/496zuar3}}.

{\let\clearpage\relax \chapter{LitQEval}\label{ch:ownApproach}}
Despite ongoing research on automatic literature query generation and related evaluations with medical datasets, such as CLEF \autocite{kanoulas2017clef, kanoulas2018clef, kanoulas2019clef} and the Collection of Seeds \autocite{Wang_2022}, the insights gained from these evaluation metrics are not particularly compelling for our use case. This limitation arises from two main factors. 

First, the CLEF and Collection of Seeds datasets are exclusively focused on medical data. Although Badami's work \autocite{badami2023adaptive} offers a more diverse dataset, it lacks a suitable evaluation metric. Their evaluation primarily aims to maximize recall, with minimal consideration for precision, as literature search queries often yield far more results than necessary, making precision a less effective measure in this context. 

A second limitation arises when recall is prioritized exclusively. For example, if we aim to train a model to generate queries that maximize recall, there is no penalty for generating overly broad queries, such as those that exploit wildcards, which could lead to an excessive number of irrelevant results.

To address these issues, we introduce a dataset structured similarly to that of Badami \autocite{badami2023adaptive} but designed to be more comprehensive and covering a wider range of research fields. Alongside this dataset, we propose new evaluation metrics that account for the inherently broad nature of literature search queries while penalizing excessively large queries. These metrics also emphasize the importance of accurately identifying core publications that are considered highly relevant within the field.

\vspace*{0.5cm}\section{Dataset}\label{sec:dataset}
The dataset we aim to create has three primary goals: First, it should encompass a wide range of randomly selected scientific research fields. Second, for each selected field, it should contain a set of highly relevant publications to serve as the CPs for evaluating additional publications found in these areas. Lastly, the data should consider different research intents, meaning publication that is considered relevant by a bibliometric analysis might not be relevant for a Systematic Literature Review (SLR) work.

Selecting new research field is straightforward; however, to avoid bias from ongoing research interests, we used ChatGPT to generate a list of scientific fields that are recent and not overly broad. For instance, a field like \textit{Artificial Intelligence} is vast, making it challenging to accurately and comprehensively identify core publications. Instead, we chose a more specific, problem-focused fields such as \textit{Drones in Agriculture}. To search for the corresponding bibliometric analysis we used the following query: \textit{<FIELD> AND ("Bibliometric" OR "Scientometric" OR "Systematic literature" OR "Most Influential" OR "Most Cited" OR "Scientific Landscape" OR "Literature Landscape" OR "Core Literature")} 


After identifying a sufficient number of diverse fields, 14 in our case, we sought to collect core publications for each field. Due to the difficulty of gathering core publications across a broad array of fields, we leveraged the bibliometrics community’s expertise. Specifically, we searched for bibliometric studies that identify the most relevant publications within each research area. For example, a bibliometric analysis of \textit{Drones in Agriculture} \autocite{Rejeb2022} lists the most cited publications from 1990 to 2021. In this case, 40 core publications were identified, which we manually located on Dimensions.ai and added to our dataset, omitting any publications not found in Dimensions.

This process was repeated across all selected research fields, resulting in a dataset comprising 14 fields, each containing 25–50 core publications, as shown in \autoref{fig:dataset-overview}. For the 
SLR data, we used previously collected data \autocite{badami2023adaptive} in the field of Software Engineering. Notably, the SLR data used here were replicated by executing the original query in Dimensions. However, only 7 out of the 10 original datasets were included, with SLR 2, 5, and 6 omitted due to extreme variations between the original datasets and the results retrieved from Dimensions. For instance, SLR 2 originally contained 8,911 candidate papers, but when executed in Dimensions, it yielded approximately 200,000. Overall, the dataset consists of 21 fields. This discrepancy is due to the fact that the search query in the original SLR work had many additional constraints along side the query string, such as the time span being limited between 2000 and 2010.


\begin{figure}
	\centering	
	\includegraphics[scale=0.75]{pics/dataset-overview.pdf}
	\caption[Dataset Overview of the Research Fields]{An overview of the dataset and the selected 21 research fields with respective core publications identified through bibliometric analyses or systematic literature review. The number in brackets following the field name on the x-axis represents the year of survey publication.}
	\label{fig:dataset-overview}
\end{figure}


\subsection{Dataset Analysis}

We recognize that potential biases may exist in our dataset due to its complete reliance on the bibliometric community for identifying core publications. This often implies that publications with higher citation counts are considered more relevant. To assess this, we analyzed the citation distribution per field, as provided by Dimensions, shown in \autoref{fig:dataset-citation}. Additionally, we examined the distribution of publication years per field, illustrating the time span considered in the bibliometric analyses, as shown in \autoref{fig:dataset-years}.  If we compare the distribution of publication years for the medical research field \textit{Cervical Myelopathy} with that of \textit{IoT in Healthcare}, both of which were published in 2018, we can observe distinct differences in the year distributions of their core publications. These variations may be attributed to factors such as the recency of the field, changes in terminology over time, or the nature of the research area, where one field may prioritize more established works while the other focuses on recent advancements.


For the evaluation pipeline that we will introduce, the embeddings of the core documents are essential for effectively assessing the search query, as detailed in \autoref{sec:eval-metrics}. To validate this approach, we examine the clustered embeddings of the titles and abstracts for each core publication, as well as the bibliometric analyses survey in which these documents were initially referenced. This enables us to assess whether core publications within each field exhibit semantic similarity while also demonstrating some degree of dissimilarity from publications in other fields. The resulting clusters, shown in \autoref{fig:dataset-clustering}, were generated using k-means clustering, where $k$ is set to the number of fields. For this, we use OpenAI's small embeddings model alongside UMAP\autocite{mcinnes2020umap} to reduce the dimensionality to 2D.

\begin{figure}
	\centering	
	\includegraphics[scale=0.6]{pics/umap_clustering.pdf}
	\caption[Core Publications Clustering]{This figure shows clusters of publication embeddings based on titles and abstracts, the ones marked with o are from the BAs and X from SLRs. Embeddings were generated with OpenAI's small model and reduced in dimensionality with UMAP, then clustered using k-means with $k=21$. Clusters group core publications by semantic similarity, with overlaps in fields like \textit{IoT in Healthcare} and \textit{AI on Edge Devices}, as well as most of the SLR topic, due to the similarity in the research field.}
	\label{fig:dataset-clustering}
\end{figure}

\section{Evaluation metrics}\label{sec:eval-metrics}
The standard evaluation metrics for query evaluation are recall and precision. We argue that while recall is of high importance, particularly within the community, precision in this context becomes less feasible. Specifically, retrieving only the exact core publications via a search query would be impractical without explicitly using DOIs to target them directly, which renders this metric largely obsolete and likely to be consistently low. However, we still aim to account for the number of matched publications when executing a search query to prevent models from exploiting overly large queries. To address this, we introduce the concept of \textit{Semantic Precision}.

The idea behind Semantic Precision is to evaluate the relevance of retrieved publications in comparison to the core publication set. If the retrieved publications are sufficiently similar to those in the core set, they are deemed to hold some relevance rather than being entirely unrelated. To achieve this, we assume that the core publications, encompass sufficient semantic breadth to gauge the quality of literature relevant to a specific field. We calculate Semantic Precision in three ways.

\begin{figure}[]
	\centering	
	\includegraphics[scale=0.4]{pics/sp_cos.pdf}
	\caption[Semantic Precision using Cosine Similarity]{This illustration demonstrates the effect of cosine similarity on randomly generated data in a 2D space. The core publications (CP) are shown in red, positioned between 0.4 and 0.9 on both the x- and y-axes. When we set the threshold to 0.946, based on the cosine similarity of the least similar core publication from the centroid, many retrieved publications on the opposite side of the spectrum are still assigned as relevant. This effect occurs because cosine similarity considers only the angle between vectors, ignoring their magnitude. In this case, this results in 48\% of the retrieved publications being considered relevant.}

	\label{fig:sp-cos}
\end{figure}

\subsubsection{Semantic Cosine Precision}

The first approach involves averaging the embeddings of the core publications. We then set an acceptance threshold based on the cosine similarity to the least similar core publication. This means that if the embedding of a retrieved publication is more similar to the center than the least similar core publication, we consider it a relevant publication, as shown in \autoref{fig:sp-cos}. To do so we define:
\begin{itemize}
	\item $ECPs$ as the embeddings of the core publications.
	\item $ERPs$ as the embeddings of the retrieved publications.
	\item $\cos(\vec{a}, \vec{b})$ as the cosine similarity between two vectors $\vec{a}$ and $\vec{b}$.
\end{itemize}

First, compute the centroid of the core publication embeddings:
\[
\vec{c}_{\text{centroid}} = \frac{1}{|ECPs|} \sum_{\vec{c_i} \in ECPs} \vec{c_i}
\]
Then, let the threshold similarity, $\theta$, be the cosine similarity of the least similar core publication to the centroid:
\[
\theta = \min_{\vec{c_i} \in CP} \cos(\vec{c}_{\text{centroid}}, \vec{c_i})
\]
Finally, Semantic Precision using cosine similarity ($SP_{cos}$) is defined as \autoref{eq:sp-cosine}, where $\mathbb{I}$ is an indicator function that equals 1 if the retrieved publication $\vec{r}$ meets the similarity criterion and 0 otherwise:
\begin{equation}\label{eq:sp-cosine}
	SP_{cos} = \frac{\sum_{\vec{p} \in \text{ERPs}} \mathbb{I} \left( \cos(\vec{c}_{\text{centroid}}, \vec{p}) \geq \theta \right)}{|\text{ERPs}|}
\end{equation}

\begin{figure}[h!]
	\centering	
	\includegraphics[scale=0.4]{pics/sp_mvee.pdf}
	\caption[Semantic Precision using MVEEE]{This illustration demonstrates the effect of using the Minimum Volume Enclosing Ellipsoid (MVEE) on randomly generated data in a 2D space. The core publications (CP) are shown in red, positioned between 0.4 and 0.9 on both the x- and y-axes. An ellipsoid is generated using MVEE to define the scope of relevant publications, ensuring that only those within the maximal angles and magnitudes of the core publications are considered relevant. In this case, this approach results in only 25\% of the retrieved publications being classified as relevant.}

	\label{fig:sp-mvee}
\end{figure}

\subsubsection{Semantic MVEE Precision}
For the second approach we omit the averaging of the embeddings and use Minimum Volume Enclosing Ellipsoid (MVEE), which creates the smallest ellipsoid that includes our CP, which we then use to determine which of the retrieved publications are relevant by checking whether they are within MMVE or not, as illustrated in \autoref{fig:sp-mvee}. This approach allows us to take into account all the dimensions by not only considering the angle but also the magnitude

The MVEE for the core publication set $CP$ is centered at $\delta$ with shape matrix $A$. To determine whether a retrieved publication $p$ is relevant, we check if it lies within the ellipsoid by testing the following condition:
\[
(\vec{p} - \delta)^T A (\vec{p} - \delta) \leq 1
\]
Semantic Precision MVEE ($SP_{MVEE}$) for this approach is then:

\begin{equation}\label{eq:sp-mvee}
	\text{SP}_{\text{MVEE}} = \frac{\sum_{\vec{p} \in \text{ERPs}} \mathbb{I} \left( (\vec{p} - \delta)^T A (\vec{p} - \delta) \leq 1 \right)}{|\text{ERPs}|}
\end{equation}

Additionally, in the evaluation also use the convex hull, which is the smallest convex set that encloses all the points by forming a polygon rather than a ellipsoid. A potential advantage of this approach is that it is more robust to outliers compared to the MVEE.

\subsubsection{Semantic Clustering Precision}

For the final semantic precision approach, we apply a simple clustering algorithm, such as k-means, on the document embeddings. The process iteratively adjusts the number of clusters \( K \), starting with \( K=2 \), and increases \( K \) until a specific condition is met. We define a threshold \( \theta \), which is between 0 and 1, that determines the stopping criterion based on the number of core publications in the smallest cluster. Specifically, we stop when the number of core publications (\( CPs \)) in the smallest cluster satisfies the condition:
\begin{equation}
CPs \text{ in cluster smallest cluster} \leq \theta \cdot \text{Maximum possible } CPs
 \end{equation}\label{eq:sp-clustering}
This ensures that the smallest cluster contains at least \( \theta \) of the core publications.
All the above semantic precision metrics aim to identify potential true positives that were initially not considered as \( CPs \). However, a key issue arises when the number of semantically relevant publications is large due to the broad scope of the initial query. 

For instance, if a query retrieves 50,000 publications, with 30,000 deemed relevant, this still poses a challenge. Screening such a large volume of documents is infeasible, making the results problematic. To address this, we introduce a decay factor to the semantic precision, defined as follows:

\begin{itemize}
	\item \( p \): Controls the initial slowness of the decay.
	\item \( q \): Controls the acceleration of the decay near the end.
	\item $\alpha$: The maximum threshold for the decay, representing the point at which the decay becomes negligible.
\end{itemize}

The decay function is expressed as:

\[
\lambda = \left(1 - \left(\frac{n_{\text{pubs}}}{\alpha}\right)^p\right)^q
\]


\begin{figure}[h!]
	\centering	
	\includegraphics[scale=0.7]{pics/decay_function.pdf}
	\caption[Decay Function for Semantic Precision]{This illustration demonstrates the effect of the decay factor, which ensures that the contribution of a large number of publications diminishes as the total count approaches the threshold. This prevents an overwhelming volume from biasing the semantic precision. For this example, we set the threshold (\( \alpha \)) to 50k, \( p=1.5 \), and \( q=10 \).}	
	\label{fig:decay-function}
\end{figure}


Now that we have metrics that can be used penalizes the model in case of generating a too broad of a query, we use can use it as a factor to calculate the F-Score, the goal of the standard F-score is to balance out between the recall and precision, but in our case we use $F_\beta$ instead, whereby the $\beta$ is the weighting factor of the recall, meaning the higher it is the more important the recall will be, in our case we set it to be 2, meaning that the recall is twice as important as the precision.
\begin{equation}\label{eq:f-beta}
F_\beta = (1 + \beta^2) \cdot \frac{\text{Precision} \cdot \text{Recall}}{(\beta^2 \cdot \text{Precision}) + \text{Recall}}
\end{equation}
For our specific case where $\beta = 2$, emphasizing the importance of recall, it is:
\[
	F_2 = 5 \cdot \frac{(\text{Precision} \cdot \lambda) \cdot \text{Recall}}{(4 \cdot \text{Precision}\cdot \lambda) + \text{Recall}}
\]

\subsection{Comparative Analysis}

To further understand the metrics and their impact on evaluating the dataset, we conduct an in-depth analysis using a randomly selected topic, \textit{Soft Robotics} and used its baseline query as a case study. First, we visualize the embeddings of the baseline and predicted queries \autoref{fig:sr-landscape}. The baseline query is the exact topic name, \textit{Soft Robotics}, while the predicted query is generated by the SQW. The embeddings are derived from the title and abstract of the retrieved publications and subsequently reduced to a 2D UMAP \autocite{mcinnes2020umap} space. It is important to note that a significant amount of information is likely lost due to the extreme dimensionality reduction from 1536 dimensions to 2.


\begin{figure}[!ht]
	\centering	
	\includegraphics[scale=0.7]{pics/sr-landscape.pdf}
	\caption[Embedding of Soft Robotics]{This figure visualizes the distribution of publications retrieved by both the baseline and predicted queries in a 2D space. The baseline query retrieved 20 core publications, whereas the predicted query retrieved 26 core publications out of a total of 36.}\label{fig:sr-landscape}
\end{figure}

\subsubsection{Semantic Cosine Precision}
At first, we test the Semantic Cosine Precision using the high-dimensional original embedding $E_o$, which was done as described in \autoref{eq:sp-cosine}. This resulted in 13,265 out of 17,573 publications being classified as relevant \autoref{fig:sr-cosine-baseline}. However, this high proportion of relevant publications appeared excessive, prompting further investigation into the threshold's effect on the number of semantically relevant documents.

Since we aim to use the $F_2$ score as our primary evaluation metric, we also factored it in the cost function which we want to maximize. The goal was to identify the optimal empirical threshold that balances the retrieval of core publications with the number of relevant publications. To achieve this, we used the inverse precision, defined as {\scriptsize$\frac{\text{Total Retrieved Publications}}{\text{Number of Relevant Publications}}$}, instead of standard precision. The results \autoref{fig:threshold-analysis} reveal that a threshold maximizing the number of retrieved core publications while minimizing false positives is approximately 0.69. This suggests that sometimes sacrificing a couple of core publications is rewarding because it allows us to reduce the total number of semantically relevant publications.

\begin{figure}
	\centering	
	\includegraphics[scale=0.6]{pics/threshold-analysis.pdf}
	\caption[Semantic Cosine Threshold: Empirical Analysis]{This figure illustrates the effect of the threshold on the $F_2$ score. As the threshold increases, the number of semantically relevant publications and core publications identified decreases. However, in some cases, such as \textit{Perovskite Solar Cells Stability}, the $F_2$ score continues to improve despite the loss of a core publication. This outcome is due to the $F_2$ score weighting recall twice as much as precision, allowing for stricter relevance criteria while sacrificing a single core publication.}\label{fig:threshold-analysis}
\end{figure}

After setting the threshold to the optimal empirical value, the Semantic Cosine Precision retrieves 19 out of the initially found 20 core publications while significantly reducing the number of semantically relevant publications by a factor of 4. This adjustment results in only 3,424 publications being identified as relevant, compared to the initial 13,265. However, this refinement comes at the cost of missing one core publication.

\subsubsection{Semantic MVEE Precision}

In contrast to Semantic Cosine Precision, we opt to use the 2D embeddings generated by UMAP $E_{umap}$, rather than the original high-dimensional embeddings, $E_o$. This decision was made because earlier evaluations of the dataset showed that the MVEE consistently classified at least 50\% of the total retrieved documents as relevant, which we believe is related to the high-dimensional nature of the embedding vectors and the fact that their norm always up to one, meaning that they have a constant magnitude.

We experiment with two enclosing shapes: the MVEE and a Convex Hull. The primary difference is that the MVEE tends to be larger due to its ellipsoidal shape, whereas the convex hull strictly bounds the points. Using the MVEE approach, 9,595 publications were identified as relevant out of the total 17,573 publications, as shown in \autoref{fig:sr-mvee-baseline}. In contrast, the convex hull, being smaller as expected, identified 7,609 publications as relevant, as illustrated in \autoref{fig:sr-hull-baseline}.

\begin{figure}
	\centering	
	\includegraphics[scale=0.6]{pics/sr-mvee-baseline.pdf}
	\caption[Semantic MVEE: Soft Robotics]{This figure shows the relevant publications identified by the MVEE. An advantage of this approach is that we always expect that all teh core publications to be included in the identified publications.}\label{fig:sr-mvee-baseline}
\end{figure}

For further evaluation, understanding the quality of the UMAP embeddings ($E_{umap}$) is crucial, as they do not retain the same level of semantic meaning as the original higher-dimensional embeddings ($E_o$). Unlike PCA, which is a linear transformation, calculating the exact semantic loss for UMAP embeddings is challenging due to its nonlinear nature. To approximate this loss, we utilized a Partial Least Squares Regression approach as outlined by Oskolkov\footnote{\url{https://towardsdatascience.com/umap-variance-explained-b0eacb5b0801}}. Based on this method, we estimated that the explained variance of the two-dimensional UMAP embeddings is only 7.15\%. This low explained variance underscores the significant reduction in information captured when transitioning from high-dimensional to two-dimensional space.


\subsubsection{Semantic Clustering Precision}

To cluster the embeddings, we use K-means for $2 \leq K \leq 100$, iteratively determining the smallest possible cluster containing at least 70\% of the core publications which is the threshold $\theta$ described in \autoref{eq:sp-clustering}. For this process, we use the high-dimensional embeddings, $E_o$, as input. Cosine similarity is used as the distance measure between points which requires input normalization, which is already done by the output of OpenAI's text-embedding-3-small model.

\begin{figure}[!hb]
	\centering	
	\includegraphics[scale=0.6]{pics/sr-clustering-baseline.pdf}
	\caption[Semantic Clustering: Soft Robotics]{This illustration shows the grouping of the embeddings $E_o$ in higher-dimensional space using K-means with $K = 0, 1, 2, 3, 4,$ and $5$, where $K=5$ is identified as the optimal solution. The spread-out nature of the clusters is uncommon in K-means but occurs here because clustering is performed in the higher-dimensional space before reducing the data to 2D using UMAP for visualization.}\label{fig:sr-clustering-baseline}
\end{figure}

The clustering results \autoref{fig:sr-clustering-baseline} indicate that the space can be divided into 5 groups. The best group has a size of 4,954 out of 17,573 publications and contains 15 out of the 20 core publications. We experimented with adjusting the threshold to match the quality of cosine similarity by increasing it to the next best solutions. At a threshold of approximately 0.75, the results included 6,861 publications with 17 core publications. At a threshold of approximately 0.85, all 17,573 publications were clustered together.

Additionally, we clustered the UMAP embeddings, $E_{umap}$, using the same thresholds (0.7, 0.75, and 0.85). These thresholds consistently yielded similar results, with 11,644 out of 17,573 publications identified as relevant and 19 out of the 20 core publications included. This consistency can serve as an indicator of the information loss incurred when using UMAP.

As mentioned in \autoref{sec:eval-metrics}, the $F_{\beta}$ score is the metric we will use for evaluation, with $\beta=2$, emphasizing recall by making it twice as important as precision. However, we extended the traditional $F_{\beta}$ score by incorporating components such as semantic precision in place of typical precision and a decay factor to account for the number of semantically relevant retrieved publications. To better understand the influence of each component, we visualize their effects in \autoref{fig:f-score}.

\begin{figure}
	\includegraphics[width=400px, height=200px]{pics/f_score.pdf}
	\caption[$F_{\beta}$ Components Analysis]{This illustration demonstrates the effect of each component in the $F_{\beta}$ score evaluation metric, where $\beta=2$ and the decay hyperparameters are set to $p=1.5$, $q=10$, and $\alpha=50000$. In the top left, we observe the impact of $\beta=2$, which ensures better scaling even with lower precision if the recall is 1. The top right plot shows how the score scales almost linearly with the precision. The bottom left and right plots depict the dampening effect on the $F_{\beta}$ score as the number of relevant publications increases, emphasizing the importance of controlling the decay to prevent inflated scores from overly large results.}\label{fig:f-score}
\end{figure}


{\let\clearpage\relax \chapter{Evaluation}\label{ch:eval}}
In this section, we evaluate the performance of literature search queries based on the introduced metrics. This evaluation serves as a foundation for developing tools that can potentially generate automatic literature search queries in the future. It is crucial to note that the objective of this evaluation is not only to assess the SQW tool itself but also to show how the defined metrics affect any arbitrarily generated literature search query. This allows us to evaluate the quality of the query regardless of the method by which it was generated.

\vspace*{.5cm}

\section{Experimental Setup}
The curated dataset is constructed using two distinct methods to identify core publications: Bibliometric Analysis (14 topics) and Systematic Literature Review (7 topics), as illustrated in \autoref{fig:dataset-overview}. For the SLRs, the original queries used by the researchers are available. Consequently, we conduct two main experiments. In both experiments we use Dimensions.ai to retrieve all required data. The retrieval process relies on their default relevance-based sorting method, which ranks publications based on the number of keyword matches between the title-abstract and the provided query.

The first experiment involves all 21 topics from both the SLRs and BAs, where we compare a baseline query against a query generated by the SQW. The baseline query consists of the exact topic name, passed into the search engine in a non-exact fashion. For instance, the query \textit{Soft Robotics} retrieves publications containing both words in their title or abstract, even if they do not appear consecutively. 

The predicted query, however, is semi-automatically generated using the SQW tool. This process begins by providing the baseline query as input, which generates a list of keywords. These keywords are then manually sorted by the author into specific or general categories, as described in \autoref{fig:sqw-stage1}. The overarching topic is derived from the topic itself; for example, in the case of \textit{Soft Robotics}, the overarching keyword \textit{Robot} is used. In some cases, the resulting queries produced excessively large results (>100k publications). To address this, keywords were filtered to limit the results to a maximum of 50k publications, balancing evaluation cost and processing speed. Importantly, the baseline query is always included in the predicted query. This ensures that recall is at least as high for the predicted query as for the baseline, making the primary goal of the evaluation to determine whether the expanded query retrieves more core publications than the baseline without becoming overly general by retrieving irrelevant publications.

The second experiment focuses exclusively on the 7 SLR topics. It uses the exact baseline queries and results from the first experiment but compares them to SLR queries manually crafted by experts in the field rather than those generated by the SQW. These SLR queries are designed with well-defined research questions aimed at retrieving the most relevant publications that help tackle these exact questions.

\section{Results}
Using the data from the first experiment, we computed all the metrics, namely: Cosine Precision, Clustering Precision, MVEE Precision, Hull Precision, Recall, and the F2 score for each precision metric, as shown in \autoref{fig:all-metrics-1}. When examining the precision metrics, Clustering Precision distinctly stands out due to its high value in certain cases when evaluated on the predicted query. For example, both \textit{Software Fault Prediction Metrics} and \textit{Multicore Performance Prediction} achieve a clustering precision of 1.0, which happens because we cannot segment the large cluster into smaller ones while maintaining the threshold of required CPs, meaning that the smallest possible cluster is in fact the full space.

Low recall can also be an issue when computing Hull, MVEE, and Clustering Precision. For example, this issue is evident in the baseline evaluation for \textit{Multicore Performance Prediction}, \textit{Nanopharmaceuticals OR Nanonutraceuticals}, \textit{Sustainable Biofuel Economy}, \textit{Drones in Agriculture}, and \textit{Resilience in Business and Management}. The precision value for MVEE and Hull are both set to 0 manually, because the minimum required points to define a plane is 3, which are not available in this case. For clustering, it is also set to 0 due to two reasons: 1) The condition $\text{CPs} < 2$ that is defined in \autoref{algo:sp-clustering},
2) The threshold $\theta = 0.7$, which means that we need at least 4 CPs in order to be able to perform the algorithm, because $ \theta * 2 = 1.4 $ and $ \theta * 3 = 2.1 $.

\begin{figure}[!h]
	\hspace*{-1cm}	
	\includegraphics[scale=0.45]{pics/all-metrics-1.pdf}
	\caption[Evaluation: Experiment 1]{This figure shows the results of the first experiment across the datasets. An issue of the MVEE, and Hull precision metrics becomes apparent in baseline query for \textit{Drones in Agriculture} and \textit{Multicore Performance Prediction}, where the value is 0. This occurs because the retrieved CPs are $<3$, which is minimum number of CPs required to define a plane. Additionally, the impact of the decay parameter is particularly evident in cases like \textit{Robotic Arthroplasty}, where the baseline F2 score is very high. Conversely, for the predicted query, which retrieves more publications but maintains the same recall, the score is significantly lower.}\label{fig:all-metrics-1}
\end{figure}

In \autoref{fig:eval1_results}, we can better interpret the results of the first experiment by examining the differences between the scores of the predicted query and the baseline. Here, positive values indicate that the predicted query performs better, while negative values show that the baseline outperforms the predicted query. 

Considering the F2 score, a notable example of the impact of overly large queries without any recall improvement is \textit{Robotic Arthroplasty}. Both the baseline and predicted queries achieved a recall of 0.957, but the expanded predicted query from the SQW retrieved significantly more results overall. Specifically, the predicted query retrieved 22,892 publications, of which only 2,834 were relevant based on cosine similarity. In contrast, the baseline query retrieved 2,151 publications, with 1,904 classified as relevant. This demonstrates how an excessively large query can dilute the precision by increasing the total number of relevant documents retrieved without improving recall.

\begin{figure}[!t]
	\hspace*{-.8cm}	
	\includegraphics[scale=0.45]{pics/eval1_results.pdf}
	\caption[Evaluation Difference: Experiment 1]{In this figure we can see the difference in values between the predicted query from the SQW and the baseline, whereby a negative value means that the baseline is better. As anticipated we at least always achieve a similar recall, but in most cases, the SQW yields better recall. However, it severely suffers in precision. When looking at the F2 value, we can see that the tool only notably outperforms the baseline on the three topics \textit{Drones in Agriculture}, \textit{Sustainable Bio Fuel Economy}, and \textit{Multicore Performance Prediction}, whereas it shows a clear disadvantage on the topics \textit{Perovskite Solar Cells Stability}, \textit{Robotic Arthroplasty}, and \textit{Cervical Myelopathy}.}
	\label{fig:eval1_results}
\end{figure}

The results of the second experiment, which compares the actual search queries used to identify the core publications in the SLRs, are shown in \autoref{fig:all-metrics-2}. Overall, the results between the two approaches are comparable. However, the SLR queries were reconstructed to fit Dimensions' query criteria and the output format of the SQW, which have led to a degradation in their quality. Notably, the field of \textit{Multicore Performance Prediction} stands out, as it has a recall of 0 for both the SLR and baseline queries. In the case of the SLR query, this is due to the fact that only the initial query, without term expansion, was accessible \autocite{Frank2017}. The raw results for the baseline, predicted, and SLR queries can respectively be found in the appendix: \autoref{fig:baseline-results}, \autoref{fig:predicted-results}, and \autoref{fig:slr-results}.  

\begin{figure}[!t]
	\hspace*{-.8cm}		
	\includegraphics[scale=0.45]{pics/all-metrics-2.pdf}
	\caption[Evaluation: Experiment 2]{This figure shows the results of the second experiment across all the SLR datasets. Surprisingly, we can see that using the SLR query does not achieve outstanding results, which is attributed to its reconstruction and adaptation to fit dimension's search engine. Notably, the SLR query used for \textit{Multicore Performance Prediction} is only partially available for public access \autocite{Frank2017}, hence the very low scores.}\label{fig:all-metrics-2}
\end{figure}

In cases where the recall of the baseline significantly outperformed the SLR, namely \textit{Software Fault Prediction Metrics}, the cosine precision was much lower. This happened due to the unwanted behavior which can be interpreted by the cosine-F2 score being 0, indicating that the recall gain was of no value due to the excessive number of irrelevant retrieved publications.

In \autoref{fig:metrics-correlation} we observe the correlation between the metrics which ultimately showing us when to use which metric. The two important values to look at are the correlation between a precision metric with the recall, as well as their respective F2 score. For example, the Cosine precision poorly correlates with the recall, which means that more CPs will further lower the acceptance of publications as relevant, yet the correlation between the Cosine F2 and precision is decent, which means that sample is no too large to be negatively effected by the decay parameter. This suggests that the more CPs we have the more robust our cosine precision metric is.

\begin{figure}[!h]	
	\centering
	\includegraphics[scale=0.45]{pics/eval2_results.pdf}
	\caption[Evaluation Difference: Experiment 2]{This figure displays the metric values difference between the original SLR query and the baseline, where a negative value indicates that the baseline performs better. Overall, both queries show similar performance, with a few notable exceptions: in \textit{Cloud Migration}, the baseline query achieves significantly higher recall, whereas in \textit{Software Fault Prediction Metrics}, the SLR query has a much higher recall. In the case of \textit{Software Defect Prediction}, although the baseline query has a slightly better recall, it also demonstrates higher precision and a better F2 score, suggesting that it retrieves more relevant publications while maintaining low sample size.}
	\label{fig:eval2_results}
\end{figure}

\begin{figure}[!h]
	\centering
	\includegraphics[scale=0.4]{pics/metrics_correlation.pdf}
	\caption[Metrics Correlation]{This figure shows the correlation between the various presented metrics. The cosine precision does not correlate well with recall, which indicates that higher recall corresponds to a robust precision metric, thereby affecting the cosine F2 score. In contrast, the Hull and MVEE precision metrics show strong correlation with recall, meaning that the number of relevant publications increases with recall, which has a negative impact their F2 scores. Clustering metrics appear to be unstable, showing no clear correlation with other factors.}
	\label{fig:metrics-correlation}
\end{figure}


In contrast Hull and MVEE precision seem to be the exact opposite, meaning that the more CPs we have the more publications we accept as relevant, which should increase their respective F2 score due to the higher precision value, but seems to negatively effect it, due to the decay factor increasing along the precision. The clustering precision there seems to be unpredictable thus the correlation between any of the other parameters is relatively low.

\section{Discussion}

The primary goal of this work was to determine the quality of literature search queries, emphasizing recall, which is widely regarded as an essential measure in the research community. However, precision has historically been less emphasized due to the intrinsic nature of literature search queries, which tend to favor comprehensiveness over specificity. By developing multiple metrics to evaluate the relevance of publications in a semantic space, we successfully integrated precision into the evaluation framework by introducing semantic precision.

To calculate semantic precision, we employed four metrics: semantic Cosine, Clustering, MVEE, and Hull precision. Each metric demonstrated distinct advantages and limitations. Initially, we hypothesized that the cosine similarity threshold should correspond to the least similar core publication. However, in certain cases, sacrificing a core publication to substantially reduce the number of irrelevant publications proved more beneficial in terms of the F2 score. Consequently, we adopted an empirical threshold estimated by maximizing the F2 score across all topics.

For Convex Hull and MVEE, we first tested their performance using the original embeddings in their high-dimensional space. However, these metrics consistently overestimated precision, often exceeding 50\%. This discrepancy is likely due to the curse of dimensionality, which complicates the construction of accurate ellipsoids or hulls in high-dimensional spaces, possibly reflecting limitations in the embedding construction process. This observation led to the decision to switch to UMAP embeddings, which offer a reduced dimensionality and improved computational feasibility. However, while UMAP embeddings show potential, the exact amount of semantic value lost compared to the original high-dimensional embeddings remains unclear.

The first experiment is as expected, the predicted query consistently achieves similar or better recall across all topics due to the inherent nature of the SQW. However, when evaluating precision, it is evident that the broader queries generated through query expansion often degrade the performance of the query. This effect is particularly visible in the F2 scores, where the increased number of irrelevant publications impacts the balance between recall and precision. While the SQW demonstrates advantages in terms of recall, its over-expansion often leads to noisy queries.

The results of the second experiment, shown in \autoref{fig:all-metrics-2}, which compare the actual search queries used to identify core publications in the SLRs, did not align with expectations, particularly in terms of recall, which was anticipated to be high. This discrepancy can be attributed to differences in how the queries were applied. The original SLR queries were designed for mixed search indices, including titles, abstracts, full text, and, in some cases, for specific fields and journals. However, for the results of this experiment to be comparable with the first experiment, the queries were adapted to fit output format of the SQW, which only produces keywords without any additional filtering. This adaptation effectively limited the scope of retrieval, making the results inherently dependent on the search engine used, in this case, Dimensions.ai.

A common issue for the Clustering, MVEE, and Hull methods lies in their dependency on recall. Semantic clustering requires at least 2 CPs for $\theta=0.5$, while this number increases depending on the value of $\theta$. Similarly, MVEE and Hull methods require at least 3 core publications to construct a plane that encloses other potentially relevant publications. In contrast, cosine similarity only requires the pre-computed mean embedding of the core publications and a similarity threshold. This independence from recall allows cosine similarity to determine relevance even when recall of the retrieved publications minimal, making it a more robust metric in cases of bad queries that retrieve no CPs.

Additionally, based on the impact of semantic precision and recall on the F2 score, as shown in \autoref{fig:metrics-correlation}, the most suitable precision should be the \textbf{Cosine Precision}. The reason is that it becomes more robust as the number CPs increases. However, for cases where the number of CPs is between 3 and 10, the Convex Hull precision could also be useful, as the number of relevant publications remains low enough to prevent significant impact on their F2 scores.






\chapter{Conclusion}\label{ch:conclusion}

\section{Summary and Contributions}
This work presents \textbf{LitQEval}, a novel framework addressing limitations in evaluating literature search query generation. Existing datasets like CLEF and Collection of Seeds focus on medical data, and while more diverse datasets like Badami’s \autocite{badami2023adaptive} exist, they lack robust evaluation metrics. The primary issues identified are the overemphasis on recall at the expense of precision and the problem of overly broad queries generating excessive irrelevant results. 

LitQEval introduces a more comprehensive dataset covering 21 diverse topics. Core publications were collected using bibliometric analyses and SLRs to ensure relevance. The dataset is validated using techniques like clustering embeddings of publication titles and abstracts, confirming semantic similarity within fields while distinguishing between different topics.

New evaluation metrics are proposed to balance recall and precision. \textit{Semantic Precision} evaluates the relevance of retrieved publications compared to core publications through four approaches: (1) cosine similarity, (2) Minimum Volume Enclosing Ellipsoid, (3) Convex Hull and (4) clustering. These metrics aim to determine relevant publications form an excessively broad queries via semantic similarity. A decay factor further adjusts precision to account for query breadth.

To balance recall and precision, the \textit{ F-$\beta$ }score emphasizes recall $\beta=2$ for evaluating queries. Comparative analyses, including a case study on "Soft Robotics," validate the metrics, revealing insights into the trade-offs between core publication retrieval and query specificity.

\section{Outlook}
This effort forms part of a broader initiative, the SQW, developed by Fraunhofer INT. The SQW aims to automate or semi-automate the creation of literature queries, expediting research on emerging topics and offering researchers a quick starting point. While this study concentrated on building a pipeline for evaluating search queries rather than testing SQW's performance, numerous opportunities remain to enhance the tool. These include using the second step, \textit{Iterative Scientific Fine-Tuning}, to refine results further. Additionally, testing optional inputs such as detailed descriptions, uploads of relevant publications, or adjusting model parameters (e.g., temperature for exploration) holds significant potential for improvement.

Beyond the SQW, the framework for evaluating query quality introduces new opportunities. For instance, it facilitates the creation of scientific chatbots capable of answering complex questions by combining standard queries, research questions, or relevant publications as inputs. Such tools could leverage cosine similarity within a large semantic space to identify and retrieve additional relevant publications effectively.

This study highlights several promising directions for future research and development. One key avenue is the refinement and broader application of the semantic precision metrics introduced here. For instance, future work could explore how to adapt these metrics to dynamic and interdisciplinary research areas where core publications may be less clearly defined. 

Lastly, the broader implications of this work in developing AI-driven research tools warrant continued exploration. From advanced literature search engines to domain-specific scientific assistants, the principles established here could inform the next generation of tools designed to augment and streamline academic research workflows.




\appendix

\chapter{Appendix}

\begin{figure}
	\centering
	\includegraphics[scale=0.35]{pics/sqw-stage1.png}
	\caption[SQW Knowledge Enrichment]{A screenshot of the SQW UI after completing the Knowledge Enrichment stage. On the left, a list of keywords is displayed alongside the number of publications associated with each keyword when used as a search term. The keywords on the right-hand side were manually categorized as general and can be roughly assessed by the number of associated publications. To narrow the scope of general keywords, we selected "agriculture" as the overarching topic. The final generated query is displayed and updated interactively as values in the transfer lists are adjusted.}

	\label{fig:sqw-stage1}
\end{figure}

\begin{figure}
	\centering	
	\includegraphics{pics/citation-distribution.pdf}
	\caption[Field citation ratio per topic]{The citation ratio per topic, showing the relative citation counts of core publications compared to the average citation frequency within their respective research fields. This illustrates how the prominence of each publication compares to typical citation levels in its field.}
	\label{fig:dataset-citation}
\end{figure}

\begin{figure}
	\centering	
	\includegraphics{pics/year-distribution.pdf}
	\caption[Distribution of publication years per topic]{The distribution of publication years for core publications across various research topics, highlighting the historical range of studies considered in the bibliometric analyses for each field. Notably, for \textit{Cervical Myelopathy}, the lower bound of publication years was set to 1980 for improved readability, although the actual range goes back to 1953.}
	\label{fig:dataset-years}
\end{figure}


\begin{figure}[!h]
	\centering	
	\includegraphics[scale=0.7]{pics/sr-cosine-baseline.pdf}
	\caption[Semantic Cosine Similarity: Soft Robotics]{This figure illustrates the publications identified as relevant using the cosine similarity measure with the threshold $\theta$ defined by \autoref{eq:sp-cosine}, which in this case was approximately 0.547. The query results included all core publications, as expected, but also classified 75\% of the total retrieved publications as relevant.}\label{fig:sr-cosine-baseline}
\end{figure}

\begin{figure}
	\centering	
	\includegraphics[scale=0.7]{pics/sr-hull-baseline.pdf}
	\caption[Semantic Cosine Threshold: Empirical Analysis]{This figure shows the relevant publications identified by the Convex Hull}\label{fig:sr-hull-baseline}
\end{figure}




\section{Further Details on Something}

\printbibliography

\begingroup
\addcontentsline{toc}{chapter}{\IfLanguageName{ngerman}{Erklärung}{Declaration}}
\chapter*{\IfLanguageName{ngerman}{Erklärung}{Declaration}}
% \thispagestyle{empty}
\IfLanguageName{ngerman}{
Ich versichere, die von mir vorgelegte Arbeit selbstständig verfasst zu haben.
Alle Stellen, die wörtlich oder sinngemäß aus veröffentlichten oder nicht veröffentlichten Arbeiten Dritter entnommen sind, habe ich als solche kenntlich gemacht.
Sämtliche Quellen und Hilfsmittel, die ich für die Arbeit benutzt habe, sind angegeben.

% adapt if tools were used
% remove if no tools have been used
(Es folgen Formulierungsbeispiele, die Sie im Sinne der Transparenz an Ihre Arbeit anpassen müssen. Über die Zulässigkeit von Hilfsmitteln sollten Sie natürlich im Vorfeld mit Ihrem Betreuer diskutiert haben.)
Insb. wurden zur Erstellung dieser Arbeit auch folgende KI Systeme eingesetzt:
\begin{itemize}
    \item ChatGPT in Version ... wurde zur initialen Ausformulierung basierend auf von mir vorgegebenen Stichpunkten in den Kapiteln ... / der gesamten Arbeit eingesetzt.
    \item ChatGPT wurde zu folgenden Themen befragt: ... / zur Generierung von Ideen bzgl. ... / Strukturierung von ... genutzt / zur Konzeption des Systems ... genutzt.
    
    Der Wortlaut der Dialoge, sowie die verwendete Version wurde im Anhang der Arbeit dokumentiert. Genutzte Passagen sind im Text als solche gekennzeichnet.
    \item ChatGPT wurde zur Erstellung von Quellcode für ... genutzt.
    
    Der Wortlaut der Dialoge, sowie die verwendete Version wurde im Anhang der Arbeit dokumentiert. Die Verwendung ist im Kopf der jeweiligen Quelldatei / Klasse / Methode / Teile kenntlich gemacht.
    \item Copilot in Version ... wurde zur Erstellung von Quellcode / Auto-Complete für ... genutzt. Die Verwendung ist im Kopf der jeweiligen Quelldatei / Klasse / Methode / Teile kenntlich gemacht.
\end{itemize}
Mir ist bewusst, dass von KI Systemen generierte Inhalte das sorgfältige wissenschaftliche Arbeiten nicht ersetzen, weshalb sämtliche derartige Inhalte durch mich kritisch überprüft und finalisiert wurden.

Die Arbeit war mit gleichem Inhalt bzw. in wesentlichen Teilen noch nicht Gegenstand einer anderen Prüfung.
}{
I declare that I have written this work by myself.
I have identified as such all passages taken verbatim or in meaning from published or unpublished works by third parties.
All sources and aids that I have used for the work are indicated.

% adapt if tools were used
% remove if no tools have been used
(Example formulations follow, which you must adapt to your work for the sake of transparency. Of course, you should have discussed about the acceptability of such aids with your supervisor in advance.)
In particular, the following AI systems were also used to create this work:
\begin{itemize}
    \item ChatGPT in version ... was used for the initial text drafting based on bullet points given by me in the chapters ... / of the entire work.
    \item ChatGPT was consulted on the following topics: ... / was used to generate ideas regarding ... / for the structuring of ... / for the conception of the system ... .
    
    The wording of the dialogs and the version used were documented in the appendix of this work. Passages used are marked as such in the text.
    \item ChatGPT was used to create source code for ... . The wording of the dialogs and the version used were documented in the appendix of this work. The use is indicated in the header of the respective source file / class / method / parts.
    \item Copilot in version ... was used to create source code / auto-complete for ... . The use is documented in the header of the respective source file / class / method / parts.
\end{itemize}
I am aware that content generated by AI systems is no substitute for careful scientific work, which is why all such generated content has been critically reviewed and finalized by me.

This work has neither been submitted with the same content nor in essential parts to any other examination authority.
}

\bigskip\bigskip
\noindent\textit{\myLocation, \myThesisSubDate}

\smallskip

\begin{flushright}
    \begin{tabular}{m{6cm}}
        \\ \hline
        \centering\myName \\
    \end{tabular}
\end{flushright}

\endgroup






\end{document}
